\documentclass[11pt, notitlepage]{report}

	\usepackage[margin=1in]{geometry}
	\usepackage{amsmath,amsthm,amssymb,amsfonts}
	\usepackage{enumitem}
	
	\newcommand{\N}{\mathbb{N}}
	\newcommand{\Z}{\mathbb{Z}}
	\newcommand{\R}{\mathbb{R}}
	\newcommand{\A}{\alpha}
	\newcommand{\ora}[1]{\overrightarrow{#1}}
	\usepackage[parfill]{parskip}
	\usepackage{mathtools}
	\newenvironment{solution}{\paragraph{\small Solution:}}{\hfill}
	\newenvironment{theorem}{\paragraph{Theorem:}}{\hfill}
	\newenvironment{definition}{\paragraph{Definition:}}{\hfill}
	\newenvironment{problem}[2][Problem]{\begin{trivlist}
	\item[\hskip \labelsep {\bfseries #1}\hskip \labelsep {\bfseries #2.}]}{\end{trivlist}}

	\usepackage{pgfplots}
	\usetikzlibrary{arrows}
	\usetikzlibrary{decorations.markings}
	\usetikzlibrary{datavisualization}
	\usetikzlibrary{datavisualization.formats.functions}
	%\usepackage{pstricks-add}

	\pgfplotsset{every axis/.append style={
	                   axis x line=middle,    % put the x axis in the middle
	                   axis y line=middle,    % put the y axis in the middle
	                   axis line style={<->,color=gray}, % arrows on the axis
	                   xlabel={$x$},          % default put x on x-axis
	                   ylabel={$y$},          % default put y on y-axis
	           }}
	\pgfplotsset{compat=1.15}
	
	\newcommand{\pgraph}[4]{
		\begin{center}
		
		\begin{tikzpicture}
		\begin{axis}[
		   trig format plots=rad,
		   axis equal,
		   grid=both
		]
		\addplot [domain=#3:#4, variable=\t, samples=50, black, decoration={
		   markings,
		   mark=between positions 0.2 and 1 step 4em with {\arrow [scale=1.5]{stealth}}
		   }, postaction=decorate]
		({#1}, {#2});
		
		\end{axis}
		\end{tikzpicture}
		
		\end{center}
	}


   \newcommand{\cgraph}[3]{
	   \begin{center}
	
	   \begin{tikzpicture}
	   \begin{axis}[
	       trig format plots=rad,
	       axis equal,
	       grid=both
	   ]
	   \addplot [domain=#2:#3, variable=\x, samples=50, black, decoration={
	       markings,
	       mark=between positions 0.2 and 1 step 4em with {\arrow [scale=1.5]{stealth}}
	       }, postaction=decorate]
	   {#1};
	
	   \end{axis}
	   \end{tikzpicture}
	
	   \end{center}
	}


	
	\makeatletter
	\newcommand*{\toccontents}{\@starttoc{toc}}
	\makeatother


\begin{document}
   \title{Math 54: Homework 3}
   \author{Abhijay Bhatnagar}
   \maketitle
   \toccontents



\setcounter{secnumdepth}{0} %% no numbering
\section{Assignment}

1.5: 1, 5, 9, 23, 24, 25, 38, 39. \\
1.7: 1, 7, 9, 11, 21, 22, 31, 32, 33, 34, 37, 38.

\newpage
\section {Section 1.5}
\subsection{Problem 1}
Determine if the system has a nontrivial solution:

\begin{align*}
	2x_1-5x_2+8x_3 &= 0 \\
	2x_1-7x_2+\phantom{8}x_3 &= 0  \\
	4x_1+2x_2+7x_3 &= 0
\end{align*} 

\begin{solution}

\[
\left(\begin{matrix}{}
  2&-5 &8 \\
  2&-7 &1 \\
  4&2 &7
\end{matrix}\right)
%
\xRightarrow{}
%
\left(\begin{matrix}{}
  2&-5 &8 \\
  0&-2 &7 \\
  0&-8 &-9
\end{matrix}\right)
%
\xRightarrow{}
%
\left(\begin{matrix}{}
  2&-5 &8 \\
  0&-2 &7 \\
  0&0 &-37
\end{matrix}\right)
\]

There is a nontrivial solution.

\end{solution}

\subsection{Problem 5}

Write the solution set of the given homogeneous system in parametric vector form.

\begin{solution}

\[
\left(\begin{matrix}{}
  1& 3& 1\\
  -4&-9 &2 \\
    &-3 &-6
\end{matrix}\right)
%
\xRightarrow{}
%
\left(\begin{matrix}{}
  1 & 3 & 1\\
  0 & 3 & 6 \\
    &-3 &-6
\end{matrix}\right)
%
\xRightarrow{}
%
\left(\begin{matrix}{}
  1 & 3 & 1\\
  0 & 1 & 2 \\
  0 &0 &0
\end{matrix}\right)
%
\xRightarrow{}
%
\left(\begin{matrix}{}
  1 & 0 & -5\\
  0 & 1 & 2 \\
  0 &0 &0
\end{matrix}\right)
\]

\begin{align*}
x_1 &= \phantom{-}5x_3 \\
x_2 &= -2x_3
\end{align*}

\[
\textbf{x}= \left(\begin{matrix}{}
  x_1\\
  x_2\\
  x_3
\end{matrix}\right)
= \left(\begin{matrix}{}
  5x_3\\
  -2x_3\\
  x_3
\end{matrix}\right)
= x_3
\left(\begin{matrix}{}
  5 \\
  -2 \\
  \phantom{-}1
\end{matrix}\right) (\text{with $x_3$ free})
\]


\end{solution}

\newpage
\subsection{Problem 9}
Describe parametric solution set.

\begin{solution}

\[\left(\begin{matrix}{}
  3& -9&6 \\
  -1&3 &-2 \\
\end{matrix}\right)=
\left(\begin{matrix}{}
  1& -3&2 \\
  0&0 &0 \\
\end{matrix}\right)
\]

\[
\textbf{x}= \left(\begin{matrix}{}
  x_1\\
  x_2\\
  x_3
\end{matrix}\right)
= \left(\begin{matrix}{}
  3x_2-2x_3\\
  x_2\\
  x_3
\end{matrix}\right)
= x_2
\left(\begin{matrix}{}
  3 \\
  1 \\
  0
\end{matrix}\right)+
x_3
\left(\begin{matrix}{}
  -2 \\
  0 \\
  1
\end{matrix}\right) (\text{with $x_2,x_3$ free})
\]

\end{solution}

\subsection{Problem 23}
True or False.

\begin{solution}
\begin{enumerate}[label=\alph*.)]
	\item True, homogenous solutions always at least have the trivial solution.
	\item False, it provides an implicit description of solution set.
	\item False, it always has the trivial solution.
	\item False, it is a line through p parallel to v.
	\item True.
\end{enumerate}
\end{solution}

\subsection{Problem 24}

True or False.

\begin{solution}

\begin{enumerate}[label=\alph*.)]
	\item False, at least one entry is nonzero
	\item True, definition of plane.
	\item True, b has to be zero.
	\item True, p is translation vector. 
	\item False, only if there is a vector solution to $Ax=b$.
\end{enumerate}

\end{solution}

\subsection{Problem 25}

Let w be any solution of Ax=b.

Show $v_h=$ w$-$p is a solution of Ax=0.

\begin{solution}

\begin{align*}
A\cdot \textbf{w} 	&= \textbf{b} 	\\
A\cdot \textbf{p} 	&= \textbf{b} 	\\
A\cdot \textbf{v}_h &= 0	    			\\
A\cdot (p+v_h) 		&= Ap + Av_h 	\\
					&= b + 0 		\\
					&= b		 		\\
\end{align*}

\end{solution}

\subsection{Problem 38}
\begin{solution}

No, an inconsistent solution implies there is a zero row. If $Ax=z$ has a solution, it has a free variable.

\end{solution}

\subsection{Problem 39}
\begin{solution}

\begin{align*}
Ax&=0\\
Au&=\sum_{i}^{n}{u_i\textbf{v}_i}=0 \\\\
A(cu)&=c\sum_{i}^{n}{u_i\textbf{v}_i}\\
&=c\cdot0\\
&=0
\end{align*}


\end{solution}

\section{Section 1.7}
\subsection{Problem 1}
Determine linear dependence
\begin{solution}
Linearly independent. It can be reduced to REF with no free variables.
\end{solution}

\subsection{Problem 7}
Determine if columns are linearly independent.
\begin{solution}
Linearly dependent. There are 4 columns and 3 entries. By Theorem 8, this is a linearly dependent set.
\end{solution}
\subsection{Problem 9}
\begin{solution}
\begin{enumerate}[label=\alph*.)]
	\item None. span$\{v_1, v_2\}$ is a line that $v_3$ isn't on.
	\item All $v_1$ and $v_2$ are linearly dependent.
\end{enumerate}
\end{solution}
\subsection{Problem 11}

\begin{solution}

\[
\left(\begin{matrix}{}
  1&-1 &4 \\
  3&-5 &7 \\
  -1&5 &h
\end{matrix}\right)
\xRightarrow{}
\left(\begin{matrix}{}
  1&-1 &4 \\
  0&-2 &-5 \\
  0&4 &h+4
\end{matrix}\right)
\xRightarrow{}
\left(\begin{matrix}{}
  1&-1 &4 \\
  0&-2 &-5 \\
  0&0 &h-6
\end{matrix}\right)
\]

Linearly dependent when $h=6$.
\end{solution}

\subsection{Problem 21}
True or False.
\begin{solution}
\begin{enumerate}[label=\alph*.)]
	\item False, it's linearly independent when it \textit{only} has the trivial solution.
	\item False, at least one vector is a linear combination of the others.
	\item True, columns > rows.
	\item True, definition of dependence.
\end{enumerate}
\end{solution}

\subsection{Problem 22}
True or False.
\begin{solution}
\begin{enumerate}[label=\alph*.)]
	\item True.
	\item False, each vector could be the zero vector.
	\item True, definition of dependence.
	\item False, not necessarily the reason for dependence.
\end{enumerate}
\end{solution}

\subsection{Problem 31}
Find nontrivial solution of Ax=0.
\begin{solution}
\[
x=\begin{bmatrix}
1\\
1\\
-1	
\end{bmatrix}
\]
\end{solution}

\subsection{Problem 32}
Find nontrivial solution of Ax=0.
\begin{solution}
\[
x=\begin{bmatrix}
1\\
2\\
-1	
\end{bmatrix}
\]
\end{solution}

\subsection{Problem 33}
\begin{solution}
True, at least one vector ($v_3$) is a linear combination of the other vectors.
\end{solution}

\subsection{Problem 34}
\begin{solution}
True, any set with the zero vector is linearly dependent.
\end{solution}

\subsection{Problem 37}
\begin{solution}
True, the linear combinations that would lead $\{v_1,v_2,v_3\}$ to be dependent would also hold for $\{v_1,v_2,v_3,v_4\}$ if you scale $v_4$ by $0$.
\end{solution}

\subsection{Problem 38}
\begin{solution}
True. Any linear combinations that would result in $\{v_1,v_2,v_3\}$ being dependent would also hold for $\{v_1,v_2,v_3,v_4\}$ if you scale $v_4$ by $0$.
\end{solution}

\end{document}
