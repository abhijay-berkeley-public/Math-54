\documentclass[11pt, notitlepage]{report}

	\usepackage[margin=1in]{geometry}
	\usepackage{amsmath,amsthm,amssymb,amsfonts}
	\usepackage{enumitem}
	\usepackage{systeme}
	\usepackage{pgfplots}

	\newcommand{\N}{\mathbb{N}}
	\newcommand{\Z}{\mathbb{Z}}
	\newcommand{\R}{\mathbb{R}}
	\newcommand{\A}{\alpha}
	\usepackage[parfill]{parskip}
	\usepackage{mathtools}
	\newenvironment{problem}{\paragraph{\small Problem:}}{\hfill}
	\newenvironment{solution}{\paragraph{Solution:}}{\hfill}
	\newenvironment{theorem}{\paragraph{Theorem:}}{\hfill}
	\newenvironment{definition}{\paragraph{Definition:}}{\hfill}
	\usetikzlibrary{arrows}
	\usetikzlibrary{decorations.markings}
	\usetikzlibrary{datavisualization}
	\usetikzlibrary{datavisualization.formats.functions}
	%\usepackage{pstricks-add}

	\pgfplotsset{every axis/.append style={
	                   axis x line=middle,    % put the x axis in the middle
	                   axis y line=middle,    % put the y axis in the middle
	                   axis line style={<->,color=gray}, % arrows on the axis
	                   xlabel={$x$},          % default put x on x-axis
	                   ylabel={$y$},          % default put y on y-axis
	           }}
	\pgfplotsset{compat=1.15}
	
	\newcommand{\pgraph}[4]{
		\begin{center}
		
		\begin{tikzpicture}
		\begin{axis}[
		   trig format plots=rad,
		   axis equal,
		   grid=both
		]
		\addplot [domain=#3:#4, variable=\t, samples=50, black, decoration={
		   markings,
		   mark=between positions 0.2 and 1 step 4em with {\arrow [scale=1.5]{stealth}}
		   }, postaction=decorate]
		({#1}, {#2});
		
		\end{axis}
		\end{tikzpicture}
		
		\end{center}
	}


   \newcommand{\cgraph}[3]{
	   \begin{center}
	
	   \begin{tikzpicture}
	   \begin{axis}[
	       trig format plots=rad,
	       axis equal,
	       grid=both
	   ]
	   \addplot [domain=#2:#3, variable=\x, samples=50, black, decoration={
	       markings,
	       mark=between positions 0.2 and 1 step 4em with {\arrow [scale=1.5]{stealth}}
	       }, postaction=decorate]
	   {#1};
	
	   \end{axis}
	   \end{tikzpicture}
	
	   \end{center}
	}


	
	\makeatletter
	\newcommand*{\toccontents}{\@starttoc{toc}}
	\makeatother


\begin{document}
   \title{Physics 5a: Homework 1}
   \author{Abhijay Bhatnagar}
   \maketitle

   \toccontents



\setcounter{secnumdepth}{0} %% no numbering
\section{Assignment}

1.2: 1, 5, 7, 11, 15, 23, 26, 30.

\newpage
\section{Problem 1.2.1}

Determine which matrices are in RREF and which are only in REF.

\begin{solution} \

RREF: a, b

REF: d
\end{solution}

\section{Problem 1.2.5}

Determine the possible echelon forms of a nonzero 2x2 matrix using Example 1 notation.

\begin{solution}
 
 \[REF:\left(\begin{matrix}{}
  \blacksquare 	& * 				\\
  0				& \blacksquare 	\\
\end{matrix}\right),
%
\]
 
 \[RREF:\left(\begin{matrix}{}
  \blacksquare 	& 0 				\\
  0				& \blacksquare 	\\
\end{matrix}\right),
%
\left(\begin{matrix}{}
  \blacksquare 	& * 				\\
  0				& 0 				\\
\end{matrix}\right),
%
\left(\begin{matrix}{}
  0				& \blacksquare 	\\
  0				& 0 				\\
\end{matrix}\right)\]
 
\end{solution}

\section{Problem 1.2.7}

Find general solutions for matrix:

\[M=\left(\begin{matrix}{}
  1&3 &4 &7  \\
  3&9 &7 &6  \\
\end{matrix}\right)\]

\begin{solution}

\[
\left(\begin{matrix}{}
  1&3 &4 &7  \\
  3&9 &7 &6  \\
\end{matrix}\right)
%
\xRightarrow{R_2=R_2-3R_1}
%
\left(\begin{matrix}{}
  1&3 &4 &7  \\
  0&0 &-5 &-15  \\
\end{matrix}\right)
%
\xRightarrow{R_2=-1/5R_2}
%
\left(\begin{matrix}{}
  1&3 &4 &7  \\
  0&0 &1 &3  \\
\end{matrix}\right)
\]


\[
Sol_M=
\systeme*{x_1=-3 x_2 - 5, x_2=$free variable$, x_3 = 3}
\]


\end{solution}

\section{Problem 1.2.11}
Find general solutions for matrix:

\[M=\left(\begin{matrix}{}
   3 &-4 &2&0  \\
  -9 &12 &-6&0  \\
  -6 & 8 &-4&0
\end{matrix}\right)\]

\begin{solution}

\[
\left(\begin{matrix}{}
   3 &-4 &2&0  \\
  -9 &12 &-6&0  \\
  -6 & 8 &-4&0
\end{matrix}\right)
%
\xRightarrow[R_3=R_3+2R_1]{R_2=R_2+3R_1}
%
\left(\begin{matrix}{}
   3 &-4 &2&0  \\
   0 & 0 & 0&0  \\
   0 & 0 & 0&0
\end{matrix}\right)
%
\xRightarrow{R_1=1/3R_1}
%
\left(\begin{matrix}{}
   1 &-4/3 &2/3&0  \\
   0 & 0 & 0&0  \\
   0 & 0 & 0&0
\end{matrix}\right)
\]

\[
Sol_M=
\systeme*{x_1= \frac{4}{3}x_2-\frac{2}{3}x_3, x_2=$free variable$, x_3 = $free variable$}
\]


\end{solution}

\section{Problem 1.2.15}

Determine if systems are consistent and if so, are unique.

\begin{solution}

\begin{enumerate}[label=\alph*)]
	\item System is consistent and unique.
	\item System is inconsistent.
\end{enumerate}
\end{solution}

\section{Problem 1.2.23}

Is a 3x5 coefficient matrix with 3 pivots consistent?

\begin{solution}

Yes, the augmented matrix will have pivots in all 3 rows \textit{before} the final column. It will just have 2 free variables, but that doesn't affect consistency.
\end{solution}

\newpage
\section{Problem 1.2.26}

Explain why a system of three equations with a corresponding coefficient matrix with three pivots is unique.

\begin{solution}
If there are three pivots in the coefficient matrix, that means all three variables have a solution. If three variables in a system of three equations all have solutions, then there are no free variables, implying that is the unique solution.
\end{solution}

\section{Problem 1.2.30}

Give an example of an inconsistent underdetermined system of two equations in three unknowns.

\begin{solution}

\[
\systeme*{x_1+x_2+x_3=3, x_1+x_2+x_3=4}
\]

\end{solution}




\end{document}
