\documentclass[11pt, notitlepage]{report}

	\usepackage[margin=1in]{geometry}
	\usepackage{amsmath,amsthm,amssymb,amsfonts}
	\usepackage{enumitem}
	\usepackage{systeme}
	\usepackage{pgfplots}

	\newcommand{\N}{\mathbb{N}}
	\newcommand{\Z}{\mathbb{Z}}
	\newcommand{\R}{\mathbb{R}}
	\newcommand{\A}{\alpha}
	\usepackage[parfill]{parskip}
	\usepackage{mathtools}
	\newenvironment{problem}{\paragraph{\small Problem:}}{\hfill}
	\newenvironment{solution}{\paragraph{Solution:}}{\hfill}
	\newenvironment{theorem}{\paragraph{Theorem:}}{\hfill}
	\newenvironment{definition}{\paragraph{Definition:}}{\hfill}
	\usetikzlibrary{arrows}
	\usetikzlibrary{decorations.markings}
	\usetikzlibrary{datavisualization}
	\usetikzlibrary{datavisualization.formats.functions}
	%\usepackage{pstricks-add}

	\pgfplotsset{every axis/.append style={
	                   axis x line=middle,    % put the x axis in the middle
	                   axis y line=middle,    % put the y axis in the middle
	                   axis line style={<->,color=gray}, % arrows on the axis
	                   xlabel={$x$},          % default put x on x-axis
	                   ylabel={$y$},          % default put y on y-axis
	           }}
	\pgfplotsset{compat=1.15}
	
	\newcommand{\pgraph}[4]{
		\begin{center}
		
		\begin{tikzpicture}
		\begin{axis}[
		   trig format plots=rad,
		   axis equal,
		   grid=both
		]
		\addplot [domain=#3:#4, variable=\t, samples=50, black, decoration={
		   markings,
		   mark=between positions 0.2 and 1 step 4em with {\arrow [scale=1.5]{stealth}}
		   }, postaction=decorate]
		({#1}, {#2});
		
		\end{axis}
		\end{tikzpicture}
		
		\end{center}
	}


   \newcommand{\cgraph}[3]{
	   \begin{center}
	
	   \begin{tikzpicture}
	   \begin{axis}[
	       trig format plots=rad,
	       axis equal,
	       grid=both
	   ]
	   \addplot [domain=#2:#3, variable=\x, samples=50, black, decoration={
	       markings,
	       mark=between positions 0.2 and 1 step 4em with {\arrow [scale=1.5]{stealth}}
	       }, postaction=decorate]
	   {#1};
	
	   \end{axis}
	   \end{tikzpicture}
	
	   \end{center}
	}


	
	\makeatletter
	\newcommand*{\toccontents}{\@starttoc{toc}}
	\makeatother


\begin{document}
   \title{Math 54: Homework 2}
   \author{Abhijay Bhatnagar}
   \maketitle

   \toccontents



\setcounter{secnumdepth}{0} %% no numbering
\section{Assignment}

1.2: 1, 5, 7, 11, 15, 23, 26, 30.\\
1.3: 1, 5, 9, 11, 14, 23, 24, 29, 32.\\
1.4: 1, 4, 11, 13, 15, 24, 25, 29, 30, 31, 34.	

\newpage
\section{Problem 1.2.1}

Determine which matrices are in RREF and which are only in REF.

\begin{solution} \

RREF: a, b

REF: d
\end{solution}

\section{Problem 1.2.5}

Determine the possible echelon forms of a nonzero 2x2 matrix using Example 1 notation.

\begin{solution}
 
 \[REF:\left(\begin{matrix}{}
  \blacksquare 	& * 				\\
  0				& \blacksquare 	\\
\end{matrix}\right),
%
\]
 
 \[RREF:\left(\begin{matrix}{}
  \blacksquare 	& 0 				\\
  0				& \blacksquare 	\\
\end{matrix}\right),
%
\left(\begin{matrix}{}
  \blacksquare 	& * 				\\
  0				& 0 				\\
\end{matrix}\right),
%
\left(\begin{matrix}{}
  0				& \blacksquare 	\\
  0				& 0 				\\
\end{matrix}\right)\]
 
\end{solution}

\section{Problem 1.2.7}

Find general solutions for matrix:

\[M=\left(\begin{matrix}{}
  1&3 &4 &7  \\
  3&9 &7 &6  \\
\end{matrix}\right)\]

\begin{solution}

\[
\left(\begin{matrix}{}
  1&3 &4 &7  \\
  3&9 &7 &6  \\
\end{matrix}\right)
%
\xRightarrow{R_2=R_2-3R_1}
%
\left(\begin{matrix}{}
  1&3 &4 &7  \\
  0&0 &-5 &-15  \\
\end{matrix}\right)
%
\xRightarrow{R_2=-1/5R_2}
%
\left(\begin{matrix}{}
  1&3 &4 &7  \\
  0&0 &1 &3  \\
\end{matrix}\right)
\]


\[
Sol_M=
\systeme*{x_1=-3 x_2 - 5, x_2=$free variable$, x_3 = 3}
\]


\end{solution}

\section{Problem 1.2.11}
Find general solutions for matrix:

\[M=\left(\begin{matrix}{}
   3 &-4 &2&0  \\
  -9 &12 &-6&0  \\
  -6 & 8 &-4&0
\end{matrix}\right)\]

\begin{solution}

\[
\left(\begin{matrix}{}
   3 &-4 &2&0  \\
  -9 &12 &-6&0  \\
  -6 & 8 &-4&0
\end{matrix}\right)
%
\xRightarrow[R_3=R_3+2R_1]{R_2=R_2+3R_1}
%
\left(\begin{matrix}{}
   3 &-4 &2&0  \\
   0 & 0 & 0&0  \\
   0 & 0 & 0&0
\end{matrix}\right)
%
\xRightarrow{R_1=1/3R_1}
%
\left(\begin{matrix}{}
   1 &-4/3 &2/3&0  \\
   0 & 0 & 0&0  \\
   0 & 0 & 0&0
\end{matrix}\right)
\]

\[
Sol_M=
\systeme*{x_1= \frac{4}{3}x_2-\frac{2}{3}x_3, x_2=$free variable$, x_3 = $free variable$}
\]


\end{solution}

\section{Problem 1.2.15}

Determine if systems are consistent and if so, are unique.

\begin{solution}

\begin{enumerate}[label=\alph*)]
	\item System is consistent and unique.
	\item System is inconsistent.
\end{enumerate}
\end{solution}

\section{Problem 1.2.23}

Is a 3x5 coefficient matrix with 3 pivots consistent?

\begin{solution}

Yes, the augmented matrix will have pivots in all 3 rows \textit{before} the final column. It will just have 2 free variables, but that doesn't affect consistency.
\end{solution}

\newpage
\section{Problem 1.2.26}

Explain why a system of three equations with a corresponding coefficient matrix with three pivots is unique.

\begin{solution}
If there are three pivots in the coefficient matrix, that means all three variables have a solution. If three variables in a system of three equations all have solutions, then there are no free variables, implying that is the unique solution.
\end{solution}

\section{Problem 1.2.30}

Give an example of an inconsistent underdetermined system of two equations in three unknowns.

\begin{solution}

\[
\systeme*{x_1+x_2+x_3=3, x_1+x_2+x_3=4}
\]

\end{solution}


\section{Problem 1.3.1}

Compute u+v and u-2v.

\[u=\left(\begin{matrix}{}
  -1 \\
  2 \\
\end{matrix}\right),
%
v=\left(\begin{matrix}{}
  -3 \\
  -1 \\
\end{matrix}\right)
\]


\begin{solution}

\[
	u+v=\left(\begin{matrix}{}
  -4 \\
  1 \\
\end{matrix}\right)\]\\
\[u-2v=\left(\begin{matrix}{}
  5 \\
  4 \\
\end{matrix}\right)
\]
\end{solution}


\newpage
\section{Problem 1.3.5}

Write equivalent system of equations.

\begin{solution}

\[
\systeme*{6x_1-3x_2=1, 
		  -x_1+4x_3=-7,
		  5x_1 =-5}
\]

\end{solution}

\section{Problem 1.3.9}

Write equivalent vector equation

\begin{solution}

\[x_1\left(\begin{matrix}{}
  0\\
  4\\
  -1
\end{matrix}\right)
+%
x_2\left(\begin{matrix}{}
  1\\
  6\\
  3
\end{matrix}\right)
+%
x_3\left(\begin{matrix}{}
  5\\
  -1\\
  -8
\end{matrix}\right)
=
\left(\begin{matrix}{}
  0\\
  0\\
  0
\end{matrix}\right)
\]

\end{solution}

\section{Problem 1.3.11}

Determine if $b$ is a linear combination of a1, a2, a3.

\begin{solution}
Yes
\end{solution}

\section{Problem 1.3.14}
Determine b is a linear combination of column vectors.
\begin{solution}
Yes
\end{solution}

\newpage
\section{Problem 1.3.23}

True or False.

\begin{solution}
\begin{enumerate}[label=\alph*)]
	\item False, (-4 3) is correct
	\item False, all of $\R^2$ is in the span. Also how do points in a plane lie on a line in 2-d space.
	\item True, technically when $c_2 = 0$
	\item True, by definition
	\item True, by definition
\end{enumerate}
\end{solution}

\section{Problem 1.3.24}

True or False.

\begin{solution}
\begin{enumerate}[label=\alph*)]
	\item True
	\item True
	\item False, the weights can be any number including 0.
	\item True
	\item True
\end{enumerate}
\end{solution}

\section{Problem 1.3.29}

Solve for center of gravity.

\begin{solution}

\[
C=\frac{1}{10}\bigg[
2\left(\begin{matrix}{}
  5\\
  -4\\
  3
\end{matrix}\right)
+
5\left(\begin{matrix}{}
  4\\
  3\\
  -2
\end{matrix}\right)
+
2\left(\begin{matrix}{}
  -4\\
  -3\\
  -1
\end{matrix}\right)
+
1\left(\begin{matrix}{}
  -9\\
  8\\
  6
\end{matrix}\right)
\bigg]
=
\left(\begin{matrix}{}
  13/10\\
  9/10\\
  0
\end{matrix}\right)
\]

\end{solution}

\section{Problem 1.3.32}
Does that vector diagram have a unique solution for the equation.
\begin{solution}
It has a solution, but it is not unique. You can get there with an infinite number of combinations of $v_2$ and $v_3$ linearly combined with $v_1$.
\end{solution}

\section{Problem 1.4.1}
Compute products.
\begin{solution} Product undefined, columns of $A$ and rows of $x$ don't match in length.
\end{solution}

\section{Problem 1.4.4}

Compute products.

\begin{solution}

\[\left(\begin{matrix}{}
  8&3&-4 \\
  5&1&2 \\
\end{matrix}\right)
\cdot
\left(\begin{matrix}{}
  1\\
  1\\
  1
\end{matrix}\right)
=
\left(\begin{matrix}{}
  7\\
  8\\
\end{matrix}\right)
\]

\end{solution}

\section{Problem 1.4.11}

Write augmented matrix, then solve.

\begin{solution}

\[\left(\begin{matrix}{}
  1&  2 &4 &-2 \\
  0&  1 &5 &2\\
  -2&-4 &-3 &9
\end{matrix}\right)
\]

System is inconsistent, no solution.

\end{solution}

\newpage
\section{Problem 1.4.13}

Is $u$ in $span\{A\}$?

\begin{solution} Yes. System $Ax=b$ is consistent.
\end{solution}

\section{Problem 1.4.15}

Show Ax=b does not have a solution for all b, then describe the set of b that does have a solution.

\begin{solution}

The second coefficient row linearly reduces to 0, so when $b_2\neq0$, the system has no solution.

Therefore, the solution space of b is:
\[
\systeme*{b_1 = \R, b_2=0}
\]
\end{solution}

\section{Problem 1.4.24}

True or False.

\begin{solution}
\begin{enumerate}[label=\alph*)]
	\item True, expanding the product of A and x yields a vector equation of linear combinations by weights.
	\item True, by definition
	\item True, by definition
	\item True, by definition
	\item False, those conditions would make the system consistent.
	\item True, if the columns do not span $\R^m$, by definition you cannot linearly combine the columns to reach certain vectors in $\R^m$.
\end{enumerate}
\end{solution}

\section{Problem 1.4.25}

\begin{solution}

\[
c_1=-3,c_2=-1,c_3=2
\]

\end{solution}

\section{Problem 1.4.29}

\begin{solution}

\[\left(\begin{matrix}{}
  1&0 &3 \\
  1&1 &4 \\
  1&3 &1
\end{matrix}\right)\]
\end{solution}

The reduced coefficient matrix will have a pivot in every column, meaning

\section{Problem 1.4.30}

Create a 3x3 matrix that does not span $\R^3$

\begin{solution}

\[M=\left(\begin{matrix}{}
  1&1 &1 \\
  2&2 &2 \\
  3&3 &3
\end{matrix}\right)\]

All linear combinations of columns of $M$ would yield a point on the line $z=y=x$.

\end{solution}

\section{Problem 1.4.31}

\begin{solution}
If there are only 2 columns corresponding to a variable space of 2, then there is no solution for any $b\in\R^3$ with 3 nonzero values.
\end{solution}

\section{Problem 1.4.34}

A is 3x3 matrix, b is vector $\in \R^3$, $Ax=b$ has unique solution. Why must the columns of A span $\R^3$.

\begin{solution}

If b has a unique solution, the reduced form of A must have pivots in all rows, which implies it can be linearly combined to any vector (x y z) in $\R^3$.

\end{solution}

\end{document}
