\documentclass[11pt, notitlepage]{report}

	\usepackage[margin=1in]{geometry}
	\usepackage{amsmath,amsthm,amssymb,amsfonts}
	\usepackage{enumitem}
	\usepackage{systeme}

	\newcommand{\N}{\mathbb{N}}
	\newcommand{\Z}{\mathbb{Z}}
	\newcommand{\R}{\mathbb{R}}
	\newcommand{\A}{\alpha}
	\newcommand{\ora}[1]{\overrightarrow{#1}}
	\newcommand{\Question}[1]{\newpage\section{#1}}
	\usepackage[parfill]{parskip}
	\usepackage{mathtools}
	\newenvironment{solution}{\paragraph{Solution:}}{\hfill}
	\newenvironment{theorem}{\paragraph{Theorem:}}{\hfill}
	\newenvironment{subtheorem}[1]{\paragraph{\small Subtheorem #1:}}{\hfill}
	\newenvironment{definition}{\paragraph{Definition:}}{\hfill}
	\newenvironment{problem}[2][Problem]{\begin{trivlist}
	\item[\hskip \labelsep {\bfseries #1}\hskip \labelsep {\bfseries #2.}]}{\end{trivlist}}
	
	\usepackage{pgfplots}
	\usetikzlibrary{arrows}
	\usetikzlibrary{decorations.markings}
	\usetikzlibrary{datavisualization}
	\usetikzlibrary{datavisualization.formats.functions}
	%\usepackage{pstricks-add}

	\pgfplotsset{every axis/.append style={
	                   axis x line=middle,    % put the x axis in the middle
	                   axis y line=middle,    % put the y axis in the middle
	                   axis line style={<->,color=gray}, % arrows on the axis
	                   xlabel={$x$},          % default put x on x-axis
	                   ylabel={$y$},          % default put y on y-axis
	           }}
	\pgfplotsset{compat=1.15}
	
	\newcommand{\pgraph}[4]{
		\begin{center}
		
		\begin{tikzpicture}
		\begin{axis}[
		   trig format plots=rad,
		   axis equal,
		   grid=both
		]
		\addplot [domain=#3:#4, variable=\t, samples=50, black, decoration={
		   markings,
		   mark=between positions 0.2 and 1 step 4em with {\arrow [scale=1.5]{stealth}}
		   }, postaction=decorate]
		({#1}, {#2});
		
		\end{axis}
		\end{tikzpicture}
		
		\end{center}
	}


   \newcommand{\cgraph}[3]{
	   \begin{center}
	
	   \begin{tikzpicture}
	   \begin{axis}[
	       trig format plots=rad,
	       axis equal,
	       grid=both
	   ]
	   \addplot [domain=#2:#3, variable=\x, samples=50, black, decoration={
	       markings,
	       mark=between positions 0.2 and 1 step 4em with {\arrow [scale=1.5]{stealth}}
	       }, postaction=decorate]
	   {#1};
	
	   \end{axis}
	   \end{tikzpicture}
	
	   \end{center}
	}


	
	\makeatletter
	\newcommand*{\toccontents}{\@starttoc{toc}}
	\makeatother


\begin{document}
   \title{Math 54: HW \#6}
   \author{Abhijay Bhatnagar}
   \maketitle
   \toccontents


%\begin{enumerate}[label=\alph*.)]
%\end{enumerate}
\setcounter{secnumdepth}{0} %% no numbering
\section{4.2}
\subsection{Problem 30}
\begin{solution}
\end{solution}
\subsection{Problem 31}
\begin{solution}
\end{solution}
\subsection{Problem 33}
\begin{solution}
\end{solution}
\subsection{Problem 35}
\begin{solution}
\end{solution}
\section{4.3}
\subsection{Problem 26}
\begin{solution}
\end{solution}
\subsection{Problem 31}
\begin{solution}
\end{solution}
\subsection{Problem 32}
\begin{solution}
\end{solution}
\subsection{Problem 33}
\begin{solution}
\end{solution}
\section{4.4}
\subsection{Problem 15}
\begin{solution}
\end{solution}
\subsection{Problem 22}
\begin{solution}
\end{solution}
\subsection{Problem 23}
\begin{solution}
\end{solution}
\subsection{Problem 24}
\begin{solution}
\end{solution}
\subsection{Problem 25}
\begin{solution}
\end{solution}
\subsection{Problem 31}
\begin{solution}
\end{solution}
\subsection{Problem 32}
\begin{solution}
\end{solution}
\section{4.5}
\subsection{Problem 9}
\begin{solution}
\end{solution}
\subsection{Problem 11}
\begin{solution}
\end{solution}
\subsection{Problem 19}
\begin{solution}
\end{solution}
\subsection{Problem 21}
\begin{solution}
\end{solution}
\subsection{Problem 23}
\begin{solution}
\end{solution}
\subsection{Problem 25}
\begin{solution}
\end{solution}
\subsection{Problem 26}
\begin{solution}
\end{solution}
\subsection{Problem 27}
\begin{solution}
\end{solution}
\subsection{Problem 29}
\begin{solution}
\end{solution}
\subsection{Problem 31}
\begin{solution}
\end{solution}
\subsection{Problem 32}
\begin{solution}
\end{solution}
\section{5.4}
\subsection{Problem 1}
\begin{solution}
\end{solution}
\subsection{Problem 3}
\begin{solution}
\end{solution}
\subsection{Problem 5}
\begin{solution}
\end{solution}
\subsection{Problem 9}
\begin{solution}
\end{solution}
\end{document}

