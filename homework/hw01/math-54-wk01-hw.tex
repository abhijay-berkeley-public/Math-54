\documentclass[11pt, notitlepage]{report}

	\usepackage[margin=1in]{geometry}
	\usepackage{amsmath,amsthm,amssymb,amsfonts}
	\usepackage{enumitem}
	
	\newcommand{\N}{\mathbb{N}}
	\newcommand{\Z}{\mathbb{Z}}
	\newcommand{\R}{\mathbb{R}}
	\newcommand{\A}{\alpha}
	\usepackage[parfill]{parskip}
	\usepackage{mathtools}
	\newenvironment{solution}{\paragraph{\small Solution:}}{\hfill}
	\newenvironment{theorem}{\paragraph{Theorem:}}{\hfill}
	\newenvironment{definition}{\paragraph{Definition:}}{\hfill}
	\newenvironment{problem}[2][Problem]{\begin{trivlist}
	\item[\hskip \labelsep {\bfseries #1}\hskip \labelsep {\bfseries #2.}]}{\end{trivlist}}
	
	\usepackage{pgfplots}
	\usetikzlibrary{arrows}
	\usetikzlibrary{decorations.markings}
	\usetikzlibrary{datavisualization}
	\usetikzlibrary{datavisualization.formats.functions}
	%\usepackage{pstricks-add}

	\pgfplotsset{every axis/.append style={
	                   axis x line=middle,    % put the x axis in the middle
	                   axis y line=middle,    % put the y axis in the middle
	                   axis line style={<->,color=gray}, % arrows on the axis
	                   xlabel={$x$},          % default put x on x-axis
	                   ylabel={$y$},          % default put y on y-axis
	           }}
	\pgfplotsset{compat=1.15}
	
	\newcommand{\pgraph}[4]{
		\begin{center}
		
		\begin{tikzpicture}
		\begin{axis}[
		   trig format plots=rad,
		   axis equal,
		   grid=both
		]
		\addplot [domain=#3:#4, variable=\t, samples=50, black, decoration={
		   markings,
		   mark=between positions 0.2 and 1 step 4em with {\arrow [scale=1.5]{stealth}}
		   }, postaction=decorate]
		({#1}, {#2});
		
		\end{axis}
		\end{tikzpicture}
		
		\end{center}
	}


   \newcommand{\cgraph}[3]{
	   \begin{center}
	
	   \begin{tikzpicture}
	   \begin{axis}[
	       trig format plots=rad,
	       axis equal,
	       grid=both
	   ]
	   \addplot [domain=#2:#3, variable=\x, samples=50, black, decoration={
	       markings,
	       mark=between positions 0.2 and 1 step 4em with {\arrow [scale=1.5]{stealth}}
	       }, postaction=decorate]
	   {#1};
	
	   \end{axis}
	   \end{tikzpicture}
	
	   \end{center}
	}


	
	\makeatletter
	\newcommand*{\toccontents}{\@starttoc{toc}}
	\makeatother


\begin{document}
   \title{Physics 5a: Homework 1}
   \author{Abhijay Bhatnagar}
   \maketitle
   \toccontents



\setcounter{secnumdepth}{0} %% no numbering
\section{Assignment}

1.1: 1, 3, 5, 7, 11, 15, 20, 23, 24, 28.	

\newpage
\section{Problem 1.}

Solve system:

\begin{align}
	x_1+5x_2&=7 \\
	-2x_1-7x_2&=-5
\end{align}


\begin{solution}

Beginning the process to convert to 

Converted to an augmented matrix:

\[\left(
\begin{matrix}{}
  1& 5& 7 \\
  -2& -7& -5 \\
\end{matrix}
\right)\]
	

Proceeding to convert to RREF

$R_2 = R_2 + 2R_1$:

\[\left(
\begin{matrix}{}
  1& 5& 7 \\
  0& 3& 9 \\
\end{matrix}
\right)\]


$R_2 = R_2 / 3$ (REF):

\[\left(
\begin{matrix}{}
  1& 5& 7 \\
  0& 1& 3 \\
\end{matrix}
\right)\]

$R_1 = R_1 - 5R_2$:


\[\left(
\begin{matrix}{}
  1& 0& -8 \\
  0& 1& 3 \\
\end{matrix}
\right)\]

$\therefore (x_1,x_2)=(-8,3)$

\end{solution}


\newpage
\section{Problem 3.}

	Find point of intersection of the lines $x_1-5x_2=1$ and $3x_1-7x_2=5$.

\begin{solution}
		Converted to an augmented matrix:

		\[\left(
		\begin{matrix}{}
		  1& 5& 7 \\
		  1& -2& -2 \\
		\end{matrix} \right)
		\xRightarrow{R_2=R_2-R_1}
		\left(\begin{matrix}{}
		  1& 5& 7 \\
		  0& -7& -9 \\
		\end{matrix}
		\right)
		\xRightarrow{R_2=-1/7R_2}
		\left(\begin{matrix}{}
		  1& 5& 7 \\
		  0& 1& 9/7 \\
		\end{matrix}
		\right)
		\xRightarrow{R_1=R_1-5R_2}
		\left(\begin{matrix}{}
		  1& 0& 4/7 \\
		  0& 1& 9/7 \\
		\end{matrix}
		\right)
		\]

$\therefore (x_1,x_2)=(4/7,9/7)$

\end{solution}

\section{Problem 5.} State in words the next two elementary row operations involved in solving a particular matrix.

\begin{enumerate}
	\item Do a row addition operation of Row 3 onto Row 2, specifically adding Row 2 three times.
	\item Do a row subtraction operation of Row 3 onto Row 1, specifically subtract it  five times.
\end{enumerate}

\section{Problem 7.}


\[\text{Row operations on:}\left(\begin{matrix}{}
		  1& 7& \phantom{-}3&-4 \\
		  & 1&-1 &\phantom{-}3\\
		  & & \phantom{-}&\phantom{-}1 \\
		  & & \phantom{-}1&-2
		\end{matrix}\right)\phantom{-----!!-}\]


\begin{enumerate}
	\item $R_2 = R_2+R_4$
		\noindent \[\left(\begin{matrix}{}
		  1& 7& 3&-4 \\
		  & 1&0 &\phantom{-}1\\
		  & & &\phantom{-}1 \\
		  & & 1&-2
		\end{matrix}\right)\]
		

	\item $R_1 = R_1-3R_4$
		\noindent \[\left(\begin{matrix}{}
		  1& 7& 0&\phantom{-}2 \\
		  & 1&0 &\phantom{-}1\\
		  & & &\phantom{-}1 \\
		  & & 1&-2
		\end{matrix}\right)\]

	\item $R_1 = R_1-7R_2$
		\noindent \[\left(\begin{matrix}{}
		  1& 0& 0&\phantom{-}16 \\
		  & 1&0 &\phantom{-}1\\
		  & & &\phantom{-}1 \\
		  & & 1&-2
		\end{matrix}\right)\]

\end{enumerate}

$R_3 \implies 0=1 \therefore$ there are no solutions.

\newpage
\section{Problem 11.}

Solve a linear system with the following augmented matrix:

\[\left(\begin{matrix}{}
  0& 1& 4&-5  \\
  1& 3& 5&-2  \\
  3& 7& 7&\phantom{-}6  \\
\end{matrix}\right)\]

\begin{solution} \
\[
\left(\begin{matrix}{}
  0& 1& 4&-5  \\
  1& 3& 5&-2  \\
  3& 7& 7&\phantom{-}6  \\
\end{matrix}\right)
\xRightarrow[R_3=R_3-3R2]{R_1,R_2=R_2,R_1}
\left(\begin{matrix}{}
  1& \phantom{-}3& 5&-2  \\
  0& \phantom{-}1& 4&-5  \\
  0& -2& -8&\phantom{-}12  \\
\end{matrix}\right)
\xRightarrow{R_3=R_3+2R_2}
\left(\begin{matrix}{}
  1& 3& 5&-2  \\
  0& 1& 4&-5  \\
  0& 0& 0&2  \\
\end{matrix}\right)\]

$R_3 \implies 0=2 \therefore$ there are no solutions.

\end{solution}



\section{Problem 15.} 

Determine if the following linear system is consistent.

\[\left(\begin{matrix}{}
  1& & 3&&2  \\
  & 1& &-3&3  \\
  & -2& 3&2&1  \\
  3&  & &7&-5  \\
\end{matrix}\right)\]

\begin{solution}
$R_3=R_3 + 2 R_2, R_4=R_4-3R_3:$
\[\left(\begin{matrix}{}
  1& & 3&&2  \\
  & 1& &-3&3  \\
  & & 3&-4&7  \\
  &  & &7&-11  \\
\end{matrix}\right)\]

Each of these rows have a pivot left of the solution column, therefore the system is consistent.
\end{solution}

\newpage

\section{Problem 20.} 

Determine value of $h$ s.t. the following matrix is the augmented matrix of a consistent linear system:
\[\left(\begin{matrix}{}
  1& h& -3  \\
  -2& 4& 6 \\
\end{matrix}\right)\]

\begin{solution}
Reducing to REF:

\[\left(\begin{matrix}{}
  1& h& -3  \\
  0& 4+2h& \phantom{-}0 \\
\end{matrix}\right)\]

The system is consistent $\iff\ 4+2h \neq 0 \iff h\neq -2$
\end{solution}


\section{Problem 23.} 

Identify statements as $true$ or $false$ and justify answer. If true, refer to definition/theorem. If false, give location of misquoted statement or counter example.

\begin{enumerate}[label=\alph*.)]
	\item \textit{Every elementary row operation is reversible.} \\
		True. Explained on bottom of page 6.
	\item \textit{A 5 x 6 matrix has six rows.} \\
		  False. Matrices are row x col, definition given on page 4.
	\item \textit{The solution set of a linear system involving variables $x_1,...,x_n$ is a list of numbers $s_1,...,s_n$ that makes each equation in the system a true statement when the values $s_1,...,s_n$ are substituted for $x_1,...,x_n$, respectively.} \\
		False. The solution set is the set of all possible solutions to a linear system. When there is only one solution, it can be considered a list of numbers in which that previous statement can be applied; however, when there are no solutions or infinitely many solutions, it is more accurately described as either no numbers, or a set of infinitely many numbers under specific conditions. Refer to 3.
	\item \textit{Two fundamental questions about a linear system involve existence and uniqueness.}\\
	True. Refer to Page 7.
\end{enumerate}

\newpage
\section{Problem 24.} 

\begin{enumerate}[label=(\alph*)]
	\item \textit{Elementary row operations on an augmented matrix never change the solution set of the associated linear system.}\\ True. Refer to page 5.

	\item \textit{Two matrices are row equivalent if they have the same number of rows.} \\ False. Row equivalence refers to interchangeability through elementary row operation, see page 6.

	\item \textit{An inconsistent system has more than one solution.} \\
     False. An inconsistent system has no solutions, see page 4.
	\item \textit{Two linear systems are equivalent if they have the same solution set.} \\ True. Refer to page 4.
\end{enumerate}

\section{Problem 28.}

Suppose $a, b, c, \text{and } d$ are constants such that $a$ is not zero and the system below is consistent for all possible values of $f$ and $g$. What can you say about the numbers $a, b, c, \text{and } d$? Justify your answer.

$ax_1 + bx_2 = f \\
 cx_1 + dx_2 = g $
 
 
 \begin{solution}\

 Augmented matrix:
 
 \[\left(\begin{matrix}{}
  a& b & f \\
  c& d & g \\
\end{matrix}\right)\]

Since $a \neq 0$, we can divide $R_1$ by $a$:

 \[\left(\begin{matrix}{}
  1& b/a & f/a \\
  c& d & g \\
\end{matrix}\right)\]

Covert to REF $R_2 = R_2 - cR_1$

 \[\left(\begin{matrix}{}
  1& b/a & f/a \\
  c& d-bc/a & g-cf/a \\
\end{matrix}\right)\]

Since the system is consistent: $d-\frac{bc}{a} \neq 0$
	
 \end{solution}
 
\end{document}