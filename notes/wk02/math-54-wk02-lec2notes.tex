\documentclass[11pt, notitlepage]{report}

	\usepackage[margin=1in]{geometry}
	\usepackage{amsmath,amsthm,amssymb,amsfonts}
	\usepackage{enumitem}
	
	\newcommand{\N}{\mathbb{N}}
	\newcommand{\Z}{\mathbb{Z}}
	\newcommand{\R}{\mathbb{R}}
	\newcommand{\A}{\alpha}
	\newcommand{\un}[1]{\text{\underline{#1}}}
	\usepackage[parfill]{parskip}
	\newenvironment{solution}{\paragraph{\small Solution:}}{\hfill}
	\newenvironment{theorem}{\paragraph{Theorem:}}{\hfill}
	\newenvironment{definition}{\paragraph{Definition:}}{\hfill}
	\newenvironment{problem}[2][Problem]{\begin{trivlist}
	\item[\hskip \labelsep {\bfseries #1}\hskip \labelsep {\bfseries #2.}]}{\end{trivlist}}
	
	\usepackage{pgfplots}
	\usetikzlibrary{arrows}
	\usetikzlibrary{decorations.markings}
	\usetikzlibrary{datavisualization}
	\usetikzlibrary{datavisualization.formats.functions}
	%\usepackage{pstricks-add}

	\pgfplotsset{every axis/.append style={
	                   axis x line=middle,    % put the x axis in the middle
	                   axis y line=middle,    % put the y axis in the middle
	                   axis line style={<->,color=gray}, % arrows on the axis
	                   xlabel={$x$},          % default put x on x-axis
	                   ylabel={$y$},          % default put y on y-axis
	           }}
	\pgfplotsset{compat=1.15}
	
	\newcommand{\pgraph}[4]{
		\begin{center}
		
		\begin{tikzpicture}
		\begin{axis}[
		   trig format plots=rad,
		   axis equal,
		   grid=both
		]
		\addplot [domain=#3:#4, variable=\t, samples=50, black, decoration={
		   markings,
		   mark=between positions 0.2 and 1 step 4em with {\arrow [scale=1.5]{stealth}}
		   }, postaction=decorate]
		({#1}, {#2});
		
		\end{axis}
		\end{tikzpicture}
		
		\end{center}
	}


   \newcommand{\cgraph}[3]{
	   \begin{center}
	
	   \begin{tikzpicture}
	   \begin{axis}[
	       trig format plots=rad,
	       axis equal,
	       grid=both
	   ]
	   \addplot [domain=#2:#3, variable=\x, samples=50, black, decoration={
	       markings,
	       mark=between positions 0.2 and 1 step 4em with {\arrow [scale=1.5]{stealth}}
	       }, postaction=decorate]
	   {#1};
	
	   \end{axis}
	   \end{tikzpicture}
	
	   \end{center}
	}


	
	\makeatletter
	\newcommand*{\toccontents}{\@starttoc{toc}}
	\makeatother


\begin{document}
   \title{Physics 5a: Homework 1}
   \author{Abhijay Bhatnagar}
   \maketitle
   \toccontents



\setcounter{secnumdepth}{0} %% no numbering
\section{Continued from last week}

%\subsection{Reduced Row Echelon Form}

\section{New Material}
\subsection{Linear transformations}

For \[y = \left(\begin{matrix}{}
  1 \\
  2 \\
  3
\end{matrix}\right),
v_1 = \left(\begin{matrix}{}
  2 \\
  1 \\
  0
\end{matrix}\right),
v_2 = \left(\begin{matrix}{}
  -2 \\
  2 \\
  6
\end{matrix}\right)
\]


Is $y$ a linear combination of  $v_1$ and $v_2$? We can use RREF to find whether there are scalars $c_1, c_2$ s.t. c$v_1$ + c$v_2$ = $y$.


\begin{definition}
	Given $v_1,...v_n \in \R^m$, the span of $v_1,...,v_n$ is denoted $span\{v_1,...,v_n\}  \in \R^m$, which is the set of all linear combinations of $v_1,...,v_n$.
\end{definition}

\textbf{e.g.}: $span\{\un{0}\} = \{\un{0}\}$ \\
\textbf{e.g. 2}: $span\{\left(\begin{matrix}{}
						  1 \\
						  1\end{matrix}\right)\} = cv: c \in \R^n$ \\
\textbf{e.g. 3}: 

  	  	$v_1 = \left(\begin{matrix}{}
				  1 \\
				  1\end{matrix}\right), v_2 = \left(\begin{matrix}{}
				  1 \\
				  0\end{matrix}\right)$


  	$span\{v_1, v_2\} = \{c_1v_1 + c_2v_2 : c_1,c_2 \in \R\} =\R^2$

\begin{proof} \

	Goal: $span\{v_1, v_2\}=\R^2 \iff$ \\ 
	For every b in R2, b is a linear comb of $v_1, v_2 \iff$ \\
	for ever b in R2, RREF (1,1,b1),(0,1,b2) is consistent.
	
	b=(b1),(b2)
\end{proof}

\textbf{e.g. 4}

  	  	$v_1 = \left(\begin{matrix}{}
				  1 \\
				  1 \\
				  0 
				  \end{matrix}\right), 
				  v_2 = \left(\begin{matrix}{}
				  0 \\
				  1 \\
				  1
				  \end{matrix}\right)$ \\
				  
  	$span\{v_1, v_2\} = \{c_1v_1 + c_2v_2 : c_1,c_2 \in \R\} =\R^2$\\
  	Intuitively, this is all points reachable by moving along v1, v2 (2d plane containing the vectors) \\
  	
  	Question: Why is the span $span\{v_1, v_2\} \notin R^3$?
  	
  	A1.) Cross product not in span
  	
  	A2.) No Vector perpendicular to plane of span is in span
  	
  	A3.) etc...
  	
  	Linear Algebra Answer: $b\in span\{v_1, v_2\} \iff$ \[
\begin{matrix}{}
  1&0 &b_1 \\
  1&0 &b_2 \\
  0&0 &b_3
\end{matrix}
\] is consistent.

	Using RREF, there are 3 cases:
	
	\begin{enumerate}
		\item no pivots
		\item 1 pivot, top right
		\item 2 pivots
	\end{enumerate}
	
	Since b3 is an undefined variable, we can define it as nonzero in all three cases so that the RREF is inconsistent, which means no matter what $b_1, b_2$ are, we can find a $b_3$ s.t. $\un{b}$ is a vector that is not in the span. 

\begin{theorem}{Equivalence Theorem}

	$v_1,..., v_n$ spans $\R^m \iff x_1 v_1+...+x_n v_n = b$ has solution for every $b \in R^m$.
	
	$\iff$
	
	\[
	\begin{matrix}{}
	  1&... &1 &  \\
	  v_1&... &v_n & b \\
	  1&... &1 & 
	\end{matrix}
	\]is consistent for every b in Rm
	
	$\iff$
	
	\newpage
	\{A, B\} has a pivot in every row of 
	A =\[
	\begin{matrix}{}
	  1&... &1 \\
	  v_1&... &v_n \\
	  1&... &1
	\end{matrix}
	\], aka need to block last augmented column with a pivot in every row
	
	
	
\end{theorem}

\textbf{Corrolary} If v1 to vn in rmr and p<m then span v to vp not in Rm

A = 1 and v and 1 coeffeciient ma



\section{Matrix Notation}

Matrix vector product ax = linear combination of columns of A with coefficients of X1...Xn
  	
  	
  	
	

\end{document}